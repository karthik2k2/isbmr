\documentclass[english]{beamer} %,handout
\usepackage{amsmath}
\usepackage{graphicx}

\makeatletter

\usepackage{listings}
\usetheme{Boadilla}

\setbeamercovered{transparent}

\usecolortheme{purdue}

\usepackage{babel}

\begin{document}

\title[TIO]{Tumor Induced Osteomalacia}

\author{Karthik Balachandran}
\institute[]
{
  Department of Endocrinology\\
  Jawaharlal Institute of Postgraduate Medical Education and Research \\
  \texttt{ \textcolor{teal}{karthik.b@jipmer.edu.in}}
}
\date{}

\begin{frame}
  \titlepage
  \vspace{-30pt}
  \begin{center}
    \includegraphics[scale=0.4]{logo}
  \end{center}
\end{frame}

\newcommand{\Gaussian}{\rput(0,-0.35){\psset{yunit=0.8cm,xunit=0.3}
     \psGauss[linecolor=red, linewidth=0.8pt, sigma=0.5]{-1.5}{1.5}}}
\def\dedge{\ncline[linestyle=dashed]}
\def\omitnode{\Tr*[edge=\dedge]{}}

\begin{frame}[<+->]{History}
\begin{itemize}
\item Developed by Markov in 1906
\item Markov was a disciple of Chebyshev  along with Lyapunov
\item Introduced for no practical reason, except maybe to spite Nekrasov\\
due to the dispute over the Weak Law of Large Numbers
\item Within 1 year it was being used to clear up issues in thermodynamics
\end{itemize}
\end{frame}



\end{document}
